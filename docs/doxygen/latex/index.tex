The Physgame engine is an abstraction layer between less portable, less user friendly, more sophistciated libraries and the game you want to make. If we do our jobs right this will save time and effort making and porting games between a variety of platforms. If you link only against this library, not a single line of your Standard compliant C++ code should need to change between platforms. At this early stage we are proving the concept with \char`\"{}Catch!\char`\"{} our first sample game. It Currently runs on Linux and Windows with an Identical codebase, when we are done with \char`\"{}Catch!\char`\"{} We want it to have one codebase, and downloadable in the Iphone app store, the Xbox store, on the PS3, on Steam, and in a variety of linux repositories.

To get the latest news on development checkout: \href{http://gitorious.org/physgame}{\tt http://gitorious.org/physgame} Or check the webpage \href{http://www.blacktoppstudios.com}{\tt http://www.blacktoppstudios.com}\hypertarget{index_Engine}{}\section{Structure}\label{index_Engine}
\hyperlink{mainloop1}{Main Loop Flow}

\hyperlink{classphys_1_1CallBackManager}{phys::CallBackManager}

\hyperlink{classphys_1_1EventManager}{phys::EventManager}

Items in the world -\/ Actor Class \hypertarget{index_Data}{}\section{Types}\label{index_Data}
\hyperlink{classPhysVector3}{PhysVector3}

PhysEvent 